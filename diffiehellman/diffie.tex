\section{Diffie-Hellman}
\subsection{De opdracht}
De opdracht rond Diffie-Hellman bestond uit twee delen: ten eerste het berekenen van de gezamenlijke sleutel en ten tweede het achterhalen van de titels waaruit de geheime sleutels gegenereerd werden.

\subsection{Gezamenlijke sleutel: eerste pogingen}
Het snelste algoritme dat we in de les zagen om dit te ontcijferen, is index calculus. We probeerden dit dus te implementeren. Dit bleek echter een redelijk ingewikkelde zaak. Er moeten heel wat keuzes gemaakt worden tijdens het algoritme (set van priemgetallen om mee te werken, ...) wat moeilijk in code te gieten is. Daarnaast moeten er heel wat priemontbindingen berekend worden. Relatief snelle algoritmes hiervoor zijn heel ingewikkeld en vielen voor ons niet te implementeren. Applicaties die snel priemontbindingen berekenen zoals Wolfram-Alpha, gebruiken hiervoor een combinatie van algoritmes tot \'e\'en ervan een resultaat vindt. Dit soort implementaties is voor ons uiteraard onhaalbaar. Index calculus bleek voor ons bij deze opdracht geen nuttig algoritme te zijn.

We hebben ook overwogen om met een lijst van titels de sleutels proberen te achterhalen, maar dit was onbegonnen werk aangezien niets geweten was over taal, soort boek, gebruik van hoofdletters en speciale tekens, spaties, ...

\subsection{Gezamenlijke sleutel: Pohlig-Hellman}
We stapten vervolgens over op het Pohlig-Hellman algoritme. Dit is in theorie trager dan index calculus, maar veel makkelijker te implementeren. Pohlig-Hellman is namelijk veel makkelijker om volledig automatisch te laten verlopen; het bevat geen vage en lastig te implementeren stappen als "kies een aantal kleine priemgetallen". Ook hoeft slechts van \'e\'en getal de priemontbinding berekend te worden, terwijl dit bij index calculus elke stap gebeurt. We berekenden de priemontbinding van $p-1$ met Wolfram-Alpha. Gezien vele wiskundige pakketten ingebouwde functies hiervoor hebben, vonden we het niet nodig dit met eigen code te berekenen. We hadden hiervoor wel code, alleen werkte deze verschrikkelijk traag.  \\Pohlig-Hellman werkt het snelst met kleine priemfactoren. De grootste priemfactor was $60432007$, wat niet restrictief groot bleek te zijn.  \\ Het algoritme implementeerden we wel helemaal zelf in Python. We baseerden ons op een uitleg op YouTube over het algoritme \footnote{\url{https://www.youtube.com/watch?v=BXFNYVmdtJU}}. Eerst berekenden we, met inputs de gegeven $A$, $g$ en $p$, voor elke priemfactor $p_i$ (of macht van priemfactor, bij meermaals voorkomende factoren) een $a_i$ zodat $a \equiv a_i (mod\ p_i)$. Hieruit haalden we dan $a\ mod \ (p-1)$ via de Chinese reststelling. Daarna draaiden we het algoritme opnieuw met $B$ om $b$ te bepalen. Gezien voor elke $a_i$ tot $p_i$ mogelijke waardes met trial and error geprobeerd moeten worden, kan het algoritme makkelijk een tiental uur nodig hebben om een resultaat te vinden. We versnelden dit enigszins door het algoritme enkele keren naast elkaar te draaien en verschillende waarden te laten testen. Enkel $a$ of $b$ is voldoende om de gezamenlijke sleutel te bepalen, maar we bepaalden ze allebei om ons resultaat te verifi\"eren en omdat we ze toch nodig hebben voor het tweede deel van de opdracht. De sleutels zijn te vinden in de appendix.

\subsection{De titels}
Om de titels van de boeken te bekomen, volstond het niet om de gevonden sleutels om te zetten in tekst. De gebruikte sleutel kan namelijk eender welke waarde zijn die modulo $p-1$ congruent is met de gevonden sleutel. We moesten dus elke $a + k * (p-1)$ en $b + k * (p-1)$ met $k$ een natuurlijk getal beschouwen. We probeerden dit met verschillende filters op het resultaat (minimaal percentage letters, maximaal 1 speciaal teken, minimaal percentage klinkers, ...). Ook voerden we wat optimalisaties door. Sleutels met oneven lengte konden we bijvoorbeeld overslaan. Na een volle dag runtime was $k$ tot 340 miljard opgelopen, zonder een bruikbaar resultaat. Merk op dat wanneer de lengte van de mogelijke titels met 1 toeneemt, er ongeveer 100 keer zoveel mogelijkheden getest moeten worden. We staakten onze pogingen wanneer we bij titellengte 31 aanbeland waren. Deze lengte volledig testen zou namelijk een drietal weken in beslag nemen. We konden geen andere manieren bedenken om effici\"ent de titels te bekomen, en besloten er niet meer verder naar op zoek te gaan.
