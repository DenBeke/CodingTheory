
\section{Playfair}
\subsection{De opdracht}
De tweede ciphertext was versleuteld met Playfair. We moesten dus de key zien te vinden waarmee een matrix was opgesteld om digrammen te versleutelen. We hadden op dit punt al een Nederlandse en een Franse tekst ontcijferd, dus we gingen ervanuit dat dit Engels was. 
Het oplossen van Playfair bleek moeilijker als ADFGVX en Vigenere, na 3 gefaalde pogingen is het ons uiteindelijk gelukt.

\subsection{Poging 1: frequentieanalyse}
Onze eerste poging bestond uit een frequentieanalyse van de digrammen. We vergeleken die waarden met gekende waarden voor digrammen in het Engels die we zelf hadden berekend aan de hand van enkele Engelse teksten. Na veel puzzelwerk raakten we echter niet erg ver. Er zijn 600 digrammen in Playfair (hoewel ze niet allemaal in de ciphertext voorkwamen) en door dicht bij elkaar gelegen frequenties was het bijna onmogelijk ze juist te kiezen. We bekeken ook de frequenties van opeenvolgende digrammen samen en we hielden rekening met frequente digrammen waarvan ook het omgekeerde frequent was, maar raakten ook hiermee niet verder.

\subsection{Poging 2: slimme frequentie analyse}
Aangezien we door gewoon naar de frequenties te kijken toch 2 digrammen vrij zeker wisten ("OS" $\leftrightarrow$ "th", "OB" $\leftrightarrow$ "he"), probeerden we om gebruik te maken van de structuur van het Playfairvierkant\footnote{\url{http://www.umich.edu/~umich/fm-34-40-2/ch7.pdf}}. Dit werkte echter ook niet, omdat we geen van de andere digrammen echt zeker wisten, en diegenen die we dachten juist te gokken niet in het vierkant bleken te passen. We probeerden de ciphertext te bruteforcen met keys waarbij deze reeds gevonden digrammen klopten, maar vonden ook zo het antwoord niet.
 
\subsection{Poging 3: Hill Climb Algoritme}
Daarna gooiden we het over een andere boeg en gingen we op zoek naar een algoritme om de ciphertext volledig geautomatiseerd te kraken. Eerst maakten we een implementatie van een hill climb algoritme. We startten door uit een startset van mogelijke keys (in ons geval 100 random gekozen keys) de besten op basis van Index of Coincedence\footnote{\url{http://practicalcryptography.com/cryptanalysis/text-characterisation/index-coincidence/}}\footnote{\url{http://en.wikipedia.org/wiki/Index_of_coincidence}} te selecteren en deze te wijzigen. Die gewijzigde keys beschouwden we dan als nieuwe startset. Op een gegeven moment bereiken we een (lokaal) maximum, wat als resultaat gezien wordt. Doordat we bij het wijzigen telkens de best scorende keys uit heel wat opties selecteerden, werkte het algoritme redelijk traag of belandde het direct in een lokaal maximum.

\subsection{Poging 4: Churn Algoritme}
Daarna stootten we op het zogenaamde Churn algoritme \footnote{\url{http://www.cryptoden.com/index.php/algorithms/churn-algorithm/20-churn-algorithm}} \footnote{\url{http://s13.zetaboards.com/Crypto/topic/6781204/1/}}. Dit algoritme is gelijkaardig aan simulated annealing, alleen heel wat simpeler om te implementeren. \\ 
Elke plaintext krijgt een score toegewezen als volgt: voor elke mogelijke digram in het Engels werd de frequentie geanalyseerd. Hiervan werden telkens de log genomen om de invloed van erg grote waarden te beperken, en werden deze waarden herschaald naar 0-9. Nu is de score van de plaintext de som van de frequenties van elk van de $l-1$ digrams in de plaintext lengte $l$. Hoe hoger de score, hoe waarschijnlijker dat de tekst Engels is. In het algoritme starten we met een zogenaamde parent key (bv. gewoon het alfabet) waarmee we de ciphertext ontcijferen en een score toekennen. Daarna voeren we een kleine verandering (permutatie van letters, kolommen of rijen) door aan de parent key en noemen we het resultaat de child key. De plaintext voor de child key wordt ook ge\"evalueerd. Indien de child key beter scoorde, vervangt deze de parent key. Indien de parent key beter scoorde, wordt een willekeurig getal uit een array van 100 getallen gekozen. Indien het verschil tussen parent en child key scores minder was dan dit gekozen getal, vervangt de child key toch de parent key. Hierdoor kan het algoritme uit lokale maxima raken. Het algoritme blijft oneindig lopen en print de uitkomst van een iteratie enkel indien een nieuwe topscore bereikt is. Merk op dat de 100 getallen in de genoemde array zodanig gekozen zijn dat de kans dat child parent vervangt, gelijkaardig is aan die bij simulated annealing. Bij sommige runs van het algoritme vonden we al na een tweeduizendtal iteraties een tekst die heel erg op Engels leek. Het antwoord was niet helemaal correct gezien bij de score geen rekening gehouden werd met de meer voorkomende X bij Playfair. Het was wel dicht genoeg bij Engels dat we de laatste aanpassingen handmatig konden doorvoeren. We vonden als key "A brief history of time", met als plaintext het begin van "A Brief History of Time: From the Big Bang to Black Holes" door Stephen Hawking. \footnote{\url{http://www.fisica.net/relatividade/stephen_hawking_a_brief_history_of_time.pdf\#page=3}}. Merk op dat we de twee lijsten met waarden op \url{http://www.cryptoden.com/} vonden, maar de code verder helemaal zelf geschreven is en enkel gebaseerd is op de beschrijving van het algoritme.Het bijbehorende vierkant is hieronder afgebeeld.   \\
\subsection{Verbeteringen aan het algoritme}
Het algoritme in playfair.py is een licht aangepaste versie van wat we eerst gebruikten. Het geeft vaker snel een antwoord door rollbacks uit te voeren als een tijdje geen vooruitgang geboekt wordt, of zelfs helemaal opnieuw te beginnen. Ook is het scoren nu aangepast aan de X'en in Playfair waardoor exact de correcte plaintext wordt gevonden. De gevonden key matrix is niet altijd de matrix gegenereerd met "abriefhistoryoftime", maar wel altijd een correcte matrix. Gezien niet alle digrammen voorkomen, zijn meerdere correcte matrices mogelijk. \\
In de eerste twee lijntjes van het Churn algoritme wordt een vaste seed ingesteld. Deze levert al na 2406 iteraties het antwoord op. Om met een andere seed te proberen, moeten de eerste twee regels code verwijderd worden. Hierdoor kan het algoritme wel langer duren. De vari\"erende runtime is een gevolg van de niet-deterministische aard van het algoritme. Gezien de code niet concurrent is, is het aangewezen om het programma meerdere keren tegelijk te draaien om sneller tot een resultaat te komen.

\begin{figure} [h!]
\centering
\begin{tabular}{c|c|c|c|c}
A&B&R&I&E\\ \hline
F&H&S&T&O\\ \hline
Y&M&D&G&J\\ \hline
K&L&N&P&Q\\ \hline
U&V&W&X&Z\\
\end{tabular}
\caption{Playfairvierkant key "A brief history of time"}
\end{figure}

